\documentclass{article}
\usepackage{graphicx} % Required for inserting images
\usepackage{amssymb}
\usepackage[margin=3cm]{geometry}
\usepackage{amsmath}
\usepackage{amsthm}
\usepackage{enumitem}
\usepackage{accents}
\usepackage{wasysym}
\usepackage{hyperref}
\usepackage{mdframed}
\usepackage[spanish]{babel}

\renewcommand{\contentsname}{Índice de Contenidos}
\renewcommand{\tablename}{Tabla}
\renewcommand{\ker}{\text{Ker}}
\newcommand{\mcd}{\text{mcd}}
\newcommand{\car}{\text{Car}}
\newcommand{\mcm}{\text{mcm}}

\begin{document}

\begin{titlepage}
    \begin{minipage}{0.4\textwidth}
        \includegraphics[height=30px]{murcialog.png}
    \end{minipage}\hfill
    \begin{minipage}{0.55\textwidth}
        \hfill{Samuel Fernández Riquelme}

        \hfill{\today}
    \end{minipage}

    \vspace{10px}

    \hrule

    \vspace{7px}

    \begin{center}
        {\Huge \textsc{Funciones de Variable \\ \vspace{5px} Compleja}}
    \end{center}

    \vspace{10px}

    \hfill{\textit{-- You are real to me.}}

    \tableofcontents
\end{titlepage}

\newpage

\section{Ejercicios}

\begin{itemize}
    \item \textit{Sean $(a_n)_{n=1}^N$ y $(b_n)_{n=1}^N$ dos sucesiones finitas de números complejos. Denotamos mediante $B_k = \sum_{n=1}^{k} b_n$ a las sumas parciales de la serie $\sum b_n$, considerando $B_0 = 0$ por convenio. Demostrar la fórmula de \textbf{sumación por partes}}
    
    \begin{equation}
        \sum_{n=M}^{N} a_nb_n = A_NB_N - a_MB_{M-1} - \sum_{n=M}^{N}(a_{n + 1} - a_n)B_n.
        \label{eq:000}
    \end{equation}

    \vspace{7px}

    \textsc{Demostración}: La idea clave de la prueba es expresar $b_n = B_n - B_{n - 1}$ y operar con la aritmética. Como debe ser siempre, antes de continuar alentamos al lector a que emplee su propio ingenio para completar la demostración.

    Ahora sí, proseguimos (\textit{spoiler alert}). Con la modificación expuesta en el paso anterior, desarrollamos el brazo izquierdo de \eqref{eq:000} como se indica

    \begin{equation*}
        \begin{split}
            \sum_{n=M}^{N} a_nb_n &= \sum_{n=M}^{N} a_n(B_n - B_{n - 1}) \\
            &= \sum_{n=M}^{N} a_nB_n - \sum_{n=M}^{N} a_nB_{n-1} \\
            &= a_NB_N + \sum_{n=M}^{N-1} a_nB_n - \sum_{n=M-1}^{N-1} a_{n+1}B_n \hspace{20px} \text{[desplazamos índice]} \\
            &= a_NB_N - a_MB_{M-1} + \sum_{n=M}^{N-1} a_nB_n - \sum_{n=M}^{N-1} a_{n+1}B_n \\
            &= a_NB_N - a_MB_{M-1} - \sum_{n=M}^{N-1} (a_{n+1} - a_n)B_n.
        \end{split}
    \end{equation*}

    Si alguna transición entre igualdades os requiere notable dedicación no os culpo, por esta vez. Hasta luego gangster. \hfill{\textsc{Q.e.d.}}

    \vspace{12px}

    \item \textit{\textbf{Teorema de Abel.} Sea $\sum_{n=0}^{\infty} a_n$ una serie de números complejos convergente. Demostrar que}
    
    \[\lim_{r \to 1, r < 1} \sum_{n=0}^{\infty} r^na_n = \sum_{n=0}^{\infty} a_n.\]

    \vspace{7px}

    \textsc{Demostración}: Aclaramos, antes de sumergirnos en el caos, que podemos considerar siempre $0 < r < 1$ y que, para cada uno de tales $r$, la serie $\sum r^na_n$ converge absolutamente. Definimos, a continuación, los valores \[A = \sum_{n=0}^{\infty} a_n \hspace{10px} \text{y} \hspace{10px} A_k = \sum_{n=0}^{k} a_n, \hspace{10px} k \in \mathbb{N},\] y notamos que, bajo las hipótesis, se cumple que \[A = A\frac{1 - r}{1 - r} = (1 - r)A\frac{1}{1-r} = (1 - r)\sum_{n=0}^{\infty}r^nA.\]
    Por otra parte, invocando \eqref{eq:000} en la serie $\sum r^na_n$ y tomando $M = 0, N \to \infty$ no es difícil convencerse de que \[\sum_{n=0}^{\infty} r^na_n = (1 - r)\sum_{n=0}^{\infty} r^nA_n.\]

    Todas estas observaciones previas nos facilitarán ahora estimar la cantidad

    \begin{equation*}
        \begin{split}
            \left|\sum_{n=0}^{\infty} r^na_n - A\right| &= \left|(1 - r)\sum_{n=0}^{\infty} r^nA_n - (1 - r)\sum_{n=0}^{\infty} r^nA\right| \\
            &= \left|(1 - r)\sum_{n=0}^{\infty} r^n(A_n - A)\right| \\
            &\leq (1 - r)\sum_{n=0}^{\infty} r^n|A_n - A|.
        \end{split}
    \end{equation*}

    Además, dado $\varepsilon > 0$, como tenemos por construcción $(A_n)_n \to A$, sabemos que existe $N \in \mathbb{N}$ tal que $|A_n - A| < \varepsilon$ para todo $n \geq N$. En consecuencia, por un lado, nos queda 

    \begin{equation}
        (1 - r)\sum_{n=N}^{\infty} r^n|A_n - A| \leq (1 - r)\varepsilon/2\sum_{n=N}^{\infty} r^n = r^N\varepsilon/2 < \varepsilon/2,
        \label{eq:001}
    \end{equation}

    y, por otro lado, como $N$ es fijo y $\sum_{n=0}^{N-1} r^n|A_n - A|$ está acotado para todo $0 < r < 1$, tomando $\delta$ suficientemente pequeño, podemos asegurar que 

    \begin{equation}
        (1 - r)\sum_{n=0}^{N-1} r^n|A_n - A| < \varepsilon/2,
        \label{eq:002}
    \end{equation}

    siempre que $1 - \delta < r < 1$. Para finalizar, unificamos \eqref{eq:001} y \eqref{eq:002} para concluir que

    \[(1 - r)\sum_{n=0}^{\infty} r^n|A_n - A| = (1 - r)\sum_{n=0}^{N-1} r^n|A_n - A| + (1 - r)\sum_{n=N}^{\infty} r^n|A_n - A| < \varepsilon,\]

    si $1 - \delta < r < 1$. \hfill{\textsc{Q.e.d.}}
\end{itemize}

\end{document}