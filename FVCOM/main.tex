\documentclass{article}
\usepackage{graphicx} % Required for inserting images
\usepackage{amssymb}
\usepackage[margin=3cm]{geometry}
\usepackage{amsmath}
\usepackage{amsthm}
\usepackage{enumitem}
\usepackage{accents}
\usepackage{wasysym}
\usepackage{hyperref}
\usepackage{mdframed}
\usepackage[spanish]{babel}

\renewcommand{\contentsname}{Índice de Contenidos}
\renewcommand{\tablename}{Tabla}
\renewcommand{\ker}{\text{Ker}}
\newcommand{\mcd}{\text{mcd}}
\newcommand{\car}{\text{Car}}
\newcommand{\mcm}{\text{mcm}}

\begin{document}

\begin{titlepage}
    \begin{minipage}{0.4\textwidth}
        \includegraphics[height=30px]{murcialog.png}
    \end{minipage}\hfill
    \begin{minipage}{0.55\textwidth}
        \hfill{Samuel Fernández Riquelme}

        \hfill{\today}
    \end{minipage}

    \vspace{10px}

    \hrule

    \vspace{7px}

    \begin{center}
        {\Huge \textsc{Funciones de Variable \\ \vspace{5px} Compleja}}
    \end{center}

    \vspace{10px}

    \hfill{\textit{-- You are real to me.}}

    \tableofcontents
\end{titlepage}

\newpage

\section{Ejercicios}

\begin{itemize}
    \item \textit{Sean $(a_n)_{n=1}^N$ y $(b_n)_{n=1}^N$ dos sucesiones finitas de números complejos. Denotamos mediante $B_k = \sum_{n=1}^{k} b_n$ a las sumas parciales de la serie $\sum b_n$, considerando $B_0 = 0$ por convenio. Demostrar la fórmula de \textbf{sumación por partes}}
    
    \begin{equation}
        \sum_{n=M}^{N} a_nb_n = A_NB_N - a_MB_{M-1} - \sum_{n=M}^{N}(a_{n + 1} - a_n)B_n.
        \label{eq:000}
    \end{equation}

    \vspace{7px}

    \textsc{Demostración}: La idea clave de la prueba es expresar $b_n = B_n - B_{n - 1}$ y operar con la aritmética. Como debe ser siempre, antes de continuar alentamos al lector a que emplee su propio ingenio para completar la demostración.

    Ahora sí, proseguimos (\textit{spoiler alert}). Con la modificación expuesta en el paso anterior, desarrollamos el brazo izquierdo de \eqref{eq:000} como se indica

    \begin{equation*}
        \begin{split}
            \sum_{n=M}^{N} a_nb_n &= \sum_{n=M}^{N} a_n(B_n - B_{n - 1}) \\
            &= \sum_{n=M}^{N} a_nB_n - \sum_{n=M}^{N} a_nB_{n-1} \\
            &= a_NB_N + \sum_{n=M}^{N-1} a_nB_n - \sum_{n=M-1}^{N-1} a_{n+1}B_n \hspace{20px} \text{[desplazamos índice]} \\
            &= a_NB_N - a_MB_{M-1} + \sum_{n=M}^{N-1} a_nB_n - \sum_{n=M}^{N-1} a_{n+1}B_n \\
            &= a_NB_N - a_MB_{M-1} - \sum_{n=M}^{N-1} (a_{n+1} - a_n)B_n.
        \end{split}
    \end{equation*}

    Si alguna transición entre igualdades os requiere notable dedicación no os culpo, por esta vez. Hasta luego gangster. \hfill{\textsc{Q.e.d.}}

    \vspace{12px}

    \item \textit{\textbf{Teorema de Abel.} Sea $\sum_{n=0}^{\infty} a_n$ una serie de números complejos convergente. Demostrar que}
    
    \[\lim_{r \to 1, r < 1} \sum_{n=0}^{\infty} r^na_n = \sum_{n=0}^{\infty} a_n.\]

    \vspace{7px}

    \textsc{Demostración}: Aclaramos, antes de sumergirnos en el caos, que podemos considerar siempre $0 < r < 1$ y que, para cada uno de tales $r$, la serie $\sum r^na_n$ converge absolutamente. Definimos, a continuación, los valores \[A = \sum_{n=0}^{\infty} a_n \hspace{10px} \text{y} \hspace{10px} A_k = \sum_{n=0}^{k} a_n, \hspace{10px} k \in \mathbb{N},\] y notamos que, bajo las hipótesis, se cumple que \[A = A\frac{1 - r}{1 - r} = (1 - r)A\frac{1}{1-r} = (1 - r)\sum_{n=0}^{\infty}r^nA.\]
    Por otra parte, invocando \eqref{eq:000} en la serie $\sum r^na_n$ y tomando $M = 0, N \to \infty$ no es difícil convencerse de que \[\sum_{n=0}^{\infty} r^na_n = (1 - r)\sum_{n=0}^{\infty} r^nA_n.\]

    Todas estas observaciones previas nos facilitarán ahora estimar la cantidad

    \begin{equation*}
        \begin{split}
            \left|\sum_{n=0}^{\infty} r^na_n - A\right| &= \left|(1 - r)\sum_{n=0}^{\infty} r^nA_n - (1 - r)\sum_{n=0}^{\infty} r^nA\right| \\
            &= \left|(1 - r)\sum_{n=0}^{\infty} r^n(A_n - A)\right| \\
            &\leq (1 - r)\sum_{n=0}^{\infty} r^n|A_n - A|.
        \end{split}
    \end{equation*}

    Además, dado $\varepsilon > 0$, como tenemos por construcción $(A_n)_n \to A$, sabemos que existe $N \in \mathbb{N}$ tal que $|A_n - A| < \varepsilon/2$ para todo $n \geq N$. En consecuencia, por un lado, nos queda 

    \begin{equation}
        (1 - r)\sum_{n=N}^{\infty} r^n|A_n - A| \leq (1 - r)\varepsilon/2\sum_{n=N}^{\infty} r^n = r^N\varepsilon/2 < \varepsilon/2,
        \label{eq:001}
    \end{equation}

    y, por otro lado, como $N$ es fijo y $\sum_{n=0}^{N-1} r^n|A_n - A|$ está acotado en $0 < r < 1$, tomando $\delta$ suficientemente pequeño, podemos asegurar que 

    \begin{equation}
        (1 - r)\sum_{n=0}^{N-1} r^n|A_n - A| < \varepsilon/2,
        \label{eq:002}
    \end{equation}

    siempre que $1 - \delta < r < 1$. Para finalizar, unificamos \eqref{eq:001} y \eqref{eq:002} para concluir que

    \[(1 - r)\sum_{n=0}^{\infty} r^n|A_n - A| = (1 - r)\sum_{n=0}^{N-1} r^n|A_n - A| + (1 - r)\sum_{n=N}^{\infty} r^n|A_n - A| < \varepsilon,\]

    si $1 - \delta < r < 1$. \hfill{\textsc{Q.e.d.}}

    \vspace{12px}

    \item \textit{Si $P \in \mathbb{C}[X]$ es un polinomio de grado $n$ con ceros $\{a_1, ..., a_n\}$, demuestra que}
    
    \begin{equation}
        \frac{P'(z)}{P(z)} = \sum_{j=1}^{n} \frac{1}{z - a_j}
        \label{eq:003}
    \end{equation}

    \vspace{5px}

    \textsc{Demostración}: Procedemos por inducción sobre $n$ el grado del polinomio. El caso base $n = 1$ es sencillísimo. 
    
    Para el salto inductivo, suponemos que todo polinomio de grado $n$ satisface \eqref{eq:003} y consideramos $P \in \mathbb{C}[X]$ arbitrario de grado $n + 1$, con ceros $\{a_1, ..., a_n, a_{n+1}\}$. Por ser $a_{n+1}$ raíz, sabemos que $P(z) = (z - a_{n+1})Q(z)$, donde $Q$ es un polinomio complejo de grado $n$, y cuyas raices son precisamente $\{a_1, ..., a_n\}$.

    Al estudiar la derivada de $P$ obtenemos \[P'(z) = ((z - a_{n+1})Q(z))' = Q(z) + (z - a_{n+1})Q'(z),\] y al dividir por $P(z)$ y hacer una instintiva aplicación de la hipótesis de inducción, completamos la desmotración:

    \begin{equation*}
        \begin{split}
            \frac{P'(z)}{P(z)} &= \frac{Q(z) + (z - a_{n+1})Q'(z)}{(z - a_{n+1})Q(z)} \\
            &= \frac{1}{z - a_{n+1}} + \frac{Q'(z)}{Q(z)} \hspace{15px} \text{[distribuir denominador]} \\
            &= \frac{1}{z - a_{n+1}} + \sum_{j=1}^{n} \frac{1}{z - a_j} \hspace{15px} \text{[aplicamos H.I.]} \\
            &= \sum_{j=1}^{n+1} \frac{1}{z - a_j} \hspace{15px} \text{[agrupar]}
        \end{split}
    \end{equation*}

    \hfill{\textsc{Q.e.d.}}

    \vspace{12px}

    \item \textit{Demuestra que, para $|z| < 1$, se cumple}
    
    \[\frac{z}{1 - z^2} + \frac{z^2}{1 - z^4} + \cdots + \frac{z^{2^n}}{1 - z^{2^{n+1}}} + \cdots = \frac{z}{1-z}.\]

    \vspace{7px}

    \textsc{Demostración}: Fijado $n \geq 0$, vamos a estudiar la expresión 
    
    \begin{equation}
        \frac{z^{2^n}}{1-z^{2^{n+1}}},
        \label{eq:004}
    \end{equation}
    
    usando para ello la fórmula de la serie geométrica con término inicial $z^{2^n}$ y razón $z^{2^{n+1}}$, con lo que desembocamos en

    \[\frac{z^{2^n}}{1-z^{2^{n+1}}} = \sum_{m=0}^{\infty} \left(z^{2^{n+1}}\right)^mz^{2^n} = \sum_{m=0}^{\infty} z^{2^n(2m + 1)}.\]

    A continuación, centrando nuestra atención en el exponente de $z$, notamos que la correspondencia $(n, m) \mapsto 2^n(2m + 1)$ representa una biyección entre $\mathbb{N}^* \times \mathbb{N}^*$ y $\mathbb{N}$ (la prueba de este hecho se sale de nuestro foco teórico). 
    
    En consecuencia, al considerar todos los posibles $n \geq 0$ y sumar sus respectivos \eqref{eq:004}, la expresión resultante consiste en una reordenación de la serie $\sum_{k=1}^{\infty} z^k$, cuyo valor es precisamente \[\frac{z}{1-z}.\]

    Por último, como $|z| < 1$ por hipótesis, la serie $\sum_{k=1}^{\infty} z^k$ converge absolutamente y, a consecuencia de ello, toda reordenación tiene el mismo valor\footnote{Un así llamado Tom Apostol detalla rigurosamente dicha afirmación, a mí me dejáis tranquilo.}, con lo que conluimos la demostración. \hfill{\textsc{Q.e.d.}}

    \newpage

    \textit{Probar, bajo las mismas condiciones, que}

    \begin{equation}
        \frac{z}{1+z} + \frac{2z^2}{1+z^2} + \cdots + \frac{2^nz^{2^n}}{1+z^{2^n}} + \cdots = \frac{1}{1-z}.
        \label{eq:005}
    \end{equation}

    \vspace{7px}

    \textsc{Demostración}: De manera similar, fijamos $n \geq 0$ y examinamos la expresión

    \[\frac{z^{2^n}}{1+z^{2^n}} = \frac{z^{2^n}}{1 - (-z^{2^n})},\]

    donde esta vez el término inicial es $z^{2^n}$ y la razón es $-z^{2^n}$, por lo que, al desarrollar la serie, nos queda que 

    \[\frac{z^{2^n}}{1+z^{2^n}} = \sum_{m=0}^{\infty} \left(-z^{2^n}\right)^mz^{2^n} = \sum_{m=0}^{\infty} (-1)^mz^{2^n(m+1)},\]

    y, al multiplicar por $2^n$, obtenemos la igualdad 

    \begin{equation}
        \frac{2^nz^{2^n}}{1+z^{2^n}} = \sum_{m=0}^{\infty} (-1)^m2^nz^{2^n(m+1)}.
        \label{eq:006}
    \end{equation}

    Ahora, consideramos $k > 0$ arbitrario, lo factorizamos como $k = 2^\mu r$, donde $r \in \mathbb{N}$ es impar, y analizamos el coeficiente de $z^k$ en el brazo izquierdo de \eqref{eq:005}. Dividimos el estudio en dos casos:
    
    \begin{itemize}
        \item Para cuando $0 \leq n < \mu$, su correspondiente serie \eqref{eq:006} aporta exactamente un sumando acompañado de $z^k$, cuyo coeficiente es $(-1)^m2^n$. Además, como se debe cumplir $2^n(m+1) = k = 2^\mu r$ y $n < \mu$, deducimos que $m + 1$ es par y, por ende, $m$ impar. En consecuencia, inferimos que el coeficiente que acompaña a $z^k$ es \[(-1)^m2^n = -2^n.\]

        \item Para $n = \mu$, realizamos un razonamiento análogo al anterior, pero deduciendo que $m$ es par (pues $m + 1 = r$ impar), por lo que el coeficiente que acompaña a $z^k$ en este caso es \[(-1)^m2^n = 2^\mu.\]
    \end{itemize}

    Juntando estos análisis y notando que si $n > \mu$ el término $z^k$ no aparece en el respectivo \eqref{eq:006}, podemos asegurar que el coeficiente general de $z^k$ en el brazo izquierdo de \eqref{eq:005} es

    \[2^\mu - (2^0 + 2^1 + \cdots + 2^{\mu - 1}) = 2^\mu - 2^\mu + 1 = 1.\]

    Para terminar, como el argumento planteado en los párrafos superiroes es válido para todo $k \in \mathbb{N}$ concluimos que ambos lados de la igualdad \eqref{eq:005} representan, en efecto, una igualdad. \hfill{\textsc{Q.e.d.}}
\end{itemize}

\end{document}