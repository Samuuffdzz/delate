\documentclass{article}
\usepackage{graphicx} % Required for inserting images
\usepackage{amssymb}
\usepackage[margin=3cm]{geometry}
\usepackage{amsmath}
\usepackage{amsthm}
\usepackage{enumitem}
\usepackage{accents}
\usepackage{wasysym}
\usepackage{hyperref}
\usepackage{mdframed}
\usepackage[spanish]{babel}

\renewcommand{\contentsname}{Índice de Contenidos}
\renewcommand{\tablename}{Tabla}
\renewcommand{\ker}{\text{Ker}}
\newcommand{\mcd}{\text{mcd}}
\newcommand{\car}{\text{Car}}
\newcommand{\mcm}{\text{mcm}}

\begin{document}

\begin{titlepage}
    \begin{minipage}{0.4\textwidth}
        \includegraphics[height=30px]{murcialog.png}
    \end{minipage}\hfill
    \begin{minipage}{0.55\textwidth}
        \hfill{Samuel Fernández Riquelme}

        \hfill{\today}
    \end{minipage}

    \vspace{10px}

    \hrule

    \vspace{7px}

    \begin{center}
        {\Huge \textsc{Funciones de Variable \\ \vspace{5px} Compleja}}
    \end{center}

    \vspace{10px}

    \hfill{\textit{-- You are real to me.}}

    \tableofcontents
\end{titlepage}

\newpage

\section{Ejercicios}

\begin{itemize}
    \item \textit{Sean $(a_n)_{n=1}^N$ y $(b_n)_{n=1}^N$ dos sucesiones finitas de números complejos. Denotamos mediante $B_k = \sum_{n=1}^{k} b_n$ a las sumas parciales de la serie $\sum b_n$, considerando $B_0 = 0$ por convenio. Demostrar la fórmula de \textbf{sumación por partes}}
    
    \begin{equation}
        \sum_{n=M}^{N} a_nb_n = A_NB_N - a_MB_{M-1} - \sum_{n=M}^{N}(a_{n + 1} - a_n)B_n.
        \label{eq:000}
    \end{equation}

    \vspace{7px}

    \textsc{Demostración}: La idea clave de la prueba es expresar $b_n = B_n - B_{n - 1}$ y operar con la aritmética. Como debe ser siempre, antes de continuar alentamos al lector a que emplee su propio ingenio para completar la demostración.

    Ahora sí, proseguimos (\textit{spoiler alert}). Con la modificación expuesta en el paso anterior, desarrollamos el brazo izquierdo de \eqref{eq:000} como se indica

    \begin{equation*}
        \begin{split}
            \sum_{n=M}^{N} a_nb_n &= \sum_{n=M}^{N} a_n(B_n - B_{n - 1}) \\
            &= \sum_{n=M}^{N} a_nB_n - \sum_{n=M}^{N} a_nB_{n-1} \\
            &= a_NB_N + \sum_{n=M}^{N-1} a_nB_n - \sum_{n=M-1}^{N-1} a_{n+1}B_n \hspace{20px} \text{[desplazamos índice]} \\
            &= a_NB_N - a_MB_{M-1} + \sum_{n=M}^{N-1} a_nB_n - \sum_{n=M}^{N-1} a_{n+1}B_n \\
            &= a_NB_N - a_MB_{M-1} - \sum_{n=M}^{N-1} (a_{n+1} - a_n)B_n.
        \end{split}
    \end{equation*}

    Si alguna transición entre igualdades os requiere notable dedicación no os culpo, por esta vez. Hasta luego ganster. \hfill{\textsc{Q.e.d.}}

    \vspace{12px}

    \item \textit{\textbf{Teorema de Abel.} Sea $\sum_{n=1}^{\infty} a_n$ una serie de números complejos convergente. Demostrar que}
    
    \[\lim_{r \to 1, r < 1} \sum_{n=1}^{\infty} r^na_n = \sum_{n=1}^{\infty} a_n.\]

    \vspace{7px}

    \textsc{Demostración}: Déjame salir. $\hfill\square$
\end{itemize}

\end{document}