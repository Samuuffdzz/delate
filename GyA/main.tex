\documentclass{article}
\usepackage{graphicx} % Required for inserting images
\usepackage{amssymb}
\usepackage[margin=3cm]{geometry}
\usepackage{amsmath}
\usepackage{amsthm}
\usepackage{enumitem}
\usepackage{accents}
\usepackage{wasysym}
\usepackage{hyperref}
\usepackage{mdframed}

\renewcommand{\contentsname}{Índice de Contenidos}
\renewcommand{\tablename}{Tabla}
\newcommand{\mcd}{\text{mcd}}

\begin{document}

\begin{titlepage}
    \begin{center}
        {\Huge \textsc{Grupos Y Anillos}}
    \end{center}

    \vspace{10px}

    \hfill{\textit{-- Alto, policía, ha cometido usted un crimen.}}

    \hfill{\textit{-- Lo asumo.}}

    \hfill{\textit{-- Lo arresto.}}

    \tableofcontents
\end{titlepage}

\newpage

\section{Resolucion Ejercicios}

Caulquiera podrá interpretar el propósito de esta sección a partir del encabezado. Se redactaran solo aquellos ejercicios que susciten un interés especial (bajo mi criterio).

\begin{enumerate}
    \item[\textbf{1.1.7}] \textit{Demostrar que si $(X, *)$ es un monoide finito y $x \in X$ entonces las siguientes condiciones son equivalentes:}

    \begin{itemize}
        \item[\textit{(i)}] \textit{$x$ es cancelable por un lado.}
        \item[\textit{(ii)}] \textit{$x$ es cancelable.}
        \item[\textit{(iii)}] \textit{$x$ tiene simétrico por un lado.}
        \item[\textit{(iv)}] \textit{$x$ es simétrico.}
    \end{itemize}

    \vspace{7px}

    \textsc{Demostración}: El flujo de la prueba va a ser el siguiente: suponer $x$ cancelable por la izquierda y concluir que $x$ tiene simétrico por la derecha, para deducir, seguidamente, que $x$ es cancelable por la derecha. Queda a las necesidades de cada uno convencerse de la suficiencia de este argumento.

    Supongamos $x$ cancelable por la izquierda y definamos la aplicación $f : X \to X$ dada por \[f(a) = x * a.\] Bajo éstas hipótesis, $f$ es biyectiva. En efecto, para probar la inyectividad aplicamos el ser $x$ cancelable y llegamos a que \[f(a) = f(a') \Rightarrow x * a = x * a' \Rightarrow a = a'.\] Más aún, gracias a un resultado del ejercicio 1.1.2, como $X$ es un conjunto finito $f$ es además sobreyectiva (o, si me permites, biyectiva). 
    En consecuencia, para $e$ el nuetro de $X$, existe $a \in X$ cumpliendo $f(a) = e$, por lo que, en definitiva, nos queda \[x * a = f(a) = e,\] es decir, $x$ tiene simétrico por la derecha.

    Siguiendo esta línea, veamos que $x$ es cancelable por la derecha, suponiendo para ello $y, z \in X$ arbitrarios tales que $y * x = z * x$. El único paso restante es comprobar que \[y = y * e = y * x * a = z * x * a = z * e = z.\]

    \hfill{\textsc{Q.e.d.}}

    \vspace{12px}

    \item[\textbf{1.1.8}] \textit{Sea $*$ una operación en un conjunto $X$ y supongamos que $*$ tiene un neutro y tres elementos $a, b, c$ tales que $a \neq c$, $b$ es el simétrico por la izquierda de $a$ y $c$ es el simétrico por la izquierda de $b$. Demostrar que $*$ no es asociativa.}

    \vspace{7px}

    \textsc{Demostración}: Es un razonamiento directo, únicamente es necesario examinar la expresión $c * b * a$: \[(c * b) * a = e * a = a \neq c = c * e = c * (b * a).\]

    \hfill{\textsc{Q.e.d.}}

    \vspace{7px}

    \newpage

    \textit{Concluir que si $(M, *)$ es un monoide en el que todo elemento tiene simétrico por la izquierda, entonces $(M, *)$ es un grupo.}

    \vspace{7px}

    \textsc{Demostración}: Sea $a \in M$ arbitrario. Por hipótesis, $a$ tiene simétrico por la izquierda, digámosle $b$, y este, a su vez, tiene también simétrico por la izquierda $c$. No obstante, como $*$ es conmutativa, para no llegar a contradicción al aplicar el resultado anterior, se debe dar $a = c$. En consecuencia, para terminar, nos queda que
    \begin{equation*}
    \begin{split}
        a * b & = c * b = e, \\
        b * a & = e,
    \end{split}
    \end{equation*}
    por lo que $a$ es invertible. \hfill{\textsc{Q.e.d.}}

    \vspace{12px}

    \item[\textbf{1.2.2}] \textit{Sea $m \in \mathbb{Z}$. Demostrar que si $m$ no es cuadrado en $\mathbb{Z}$, entonces tampoco es un cuadrado en $\mathbb{Q}$.}

    \vspace{7px}

    \textsc{Demostración}: Nos ocuparemos, en su lugar, de probar el contrarrecíproco de la proposición del enunciado. Suponemos $m = q^2$ donde $q = a / b \in \mathbb{Q}, a, b \in \mathbb{Z}$ y $\mcd(a, b) = 1$. En consecuencia, deducimos que \[m = \frac{a^2}{b^2},\] y al despejar $a^2 = b^2m$ notamos que \[a^2 = b(bm) \Longrightarrow b \mid a^2,\] donde invocamos el Lema de Euclides para obtener $b \mid a$. No obstante, como $a$ y $b$ eran coprimos, la única causa justificada es que $b = 1$, por lo que concluimos finalmente que $m = a^2$. \hfill{\textsc{Q.e.d.}}

    \vspace{12px}

    \item[\textbf{1.3.2}] \textit{Decimos que un entero $d$ es} libre de cuadrados \textit{si $p^2$ no divide a $d$ para ningún número primo $p$ (en particular, 1 es libre de cuadrados). Demostrar que para todo $m \in \mathbb{Z}$ existe un $d \in \mathbb{Z}$ libre de cuadrados tal que $\mathbb{Q}[\sqrt{m}] = \mathbb{Q}[\sqrt{d}]$.}
    
    \vspace{7px}

    \textsc{Demostración}: Iniciamos la demostración factorizando $m = p_1^{a_1}p_2^{a_2}\cdots p_k^{a_k}$ y definiendo, a continuación, los números \[n = p_1^{b_1}p_2^{b_2}\cdots p_k^{b_k}, \hspace{10px} d = p_1^{a_1 - 2b_1}p_2^{a_2 - 2b_2}\cdots p_k^{a_k - 2b_k}, \hspace{10px} \text{con } b_i = \left\lfloor \frac{a_i}{2}\right\rfloor.\]

    Por una parte, es inmediato notar que $m = n^2d$ y, por otra parte, el método de construcción nos asegura que $0 \leq a_i - 2b_i \leq 1$, para cada $1 \leq i \leq k$, con lo que deducimos que $d$ es libre de cuadrados.

    Habiendo definido estos números auxiliares, notamos que para cada $a + b\sqrt{m} \in \mathbb{Q}[\sqrt{m}]$ se cumple \[a + b\sqrt{m} = a + b\sqrt{n^2d} = a + bn\sqrt{d} \in \mathbb{Q}[\sqrt{d}],\] con lo que inferimos que $\mathbb{Q}[\sqrt{m}] \subseteq \mathbb{Q}[\sqrt{d}]$. Recíprocamente, si $a + b\sqrt{d} \in \mathbb{Q}[\sqrt{d}]$, entonces \[a + b\sqrt{d} = a + \frac{b}{n}n\sqrt{d} = a + \frac{b}{n}\sqrt{m} \in \mathbb{Q}[\sqrt{m}],\] deduciendo finalmente que $\mathbb{Q}[\sqrt{m}] = \mathbb{Q}[\sqrt{d}]$. \hfill{\textsc{Q.e.d.}}

    \vspace{7px}

    \textit{¿Ocurre lo mismo si cambiamos $\mathbb{Q}$ por $\mathbb{Z}$?}

    \vspace{7px}

    \textsc{Solución}: Tomando $m = 12$ se puede comprobar que $\mathbb{Z}[\sqrt{d}] \not\subseteq \mathbb{Z}[\sqrt{12}]$, para todo $d$ libre de cuadrados. $\hfill\square$
\end{enumerate}

\end{document}