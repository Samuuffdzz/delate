\documentclass{article}
\usepackage{graphicx} % Required for inserting images
\usepackage{amssymb}
\usepackage[margin=3cm]{geometry}
\usepackage{amsmath}
\usepackage{amsthm}
\usepackage{enumitem}
\usepackage{accents}
\usepackage{wasysym}
\usepackage{hyperref}
\usepackage{mdframed}

\renewcommand{\contentsname}{Índice de Contenidos}

\begin{document}

\begin{titlepage}
    \begin{center}
        {\Huge \textsc{Grupos Y Anillos}}
    \end{center}

    \vspace{10px}

    \hfill{\textit{-- Alto, policía, ha cometido usted un crimen.}}

    \hfill{\textit{-- Lo asumo.}}

    \hfill{\textit{-- Lo arresto.}}

    \tableofcontents
\end{titlepage}

\newpage

\section{Resolucion Ejercicios}

Caulquiera que podrá interpretar el propósito de esta sección a partir del encabezado. Se redactaran solo aquellos ejercicios que susciten un interés especial (bajo mi criterio).

\begin{enumerate}
    \item[\textbf{1.1.7}] \textit{Demostrar que si $(X, *)$ es un monoide finito y $x \in X$ entonces las siguientes condiciones son equivalentes:}

    \begin{itemize}
        \item[\textit{(i)}] \textit{$x$ es cancelable por un lado.}
        \item[\textit{(ii)}] \textit{$x$ es cancelable.}
        \item[\textit{(iii)}] \textit{$x$ tiene simétrico por un lado.}
        \item[\textit{(iv)}] \textit{$x$ es simétrico.}
    \end{itemize}

    \vspace{7px}

    \textsc{Demostración}: El flujo de la prueba va a ser el siguiente: 
\end{enumerate}

\end{document}